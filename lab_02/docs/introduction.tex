\Introduction

Основной целью работы является ознакомление с принципами функционирования, построения и особенностями архитектуры суперскалярных конвейерных микропроцессоров. Дополнительной целью работы является знакомство с принципами проектирования и верификации сложных цифровых устройств с использованием языка описания аппаратуры SystemVerilog и ПЛИС.

Для достижения поставленных целей в настоящей лабораторной работе используется синтезируемое описание микропроцессорного ядра Taiga, реализующего систему команд RV32I семейства RISC-V. Данное описание выполнено на языке описания аппаратуры SystemVerilog.

В ходе лабораторной работы используется средство моделирования MentorGraphics Modelsim для моделирования работы исследуемого микропроцессора в процессе выполнения программы и наблюдения формы внутренних сигналов.

Для работы с проектом используется САПР Intel Quartus.
